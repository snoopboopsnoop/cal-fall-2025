\documentclass[11pt]{article}

\usepackage[utf8]{inputenc}
\usepackage{amsmath, amsthm, amssymb, amsfonts}
\usepackage{mathrsfs}
\usepackage{mathtools}
\usepackage{thmtools}
\usepackage{enumitem}
\usepackage{array}
\usepackage{subfigure}
\usepackage[all,arc]{xy}
\usepackage{tikz-cd}
\usepackage{xcolor}
\usepackage{graphicx}
\usepackage{braket}
\graphicspath{{./images/}}
\usepackage{wrapfig}
\usepackage{todonotes}
\usepackage[titletoc,toc,title]{appendix}
\usepackage[colorlinks = true,
            linkcolor = blue,
            urlcolor  = blue,
            citecolor = blue,
            anchorcolor = blue]{hyperref}
\usepackage{url}
\usepackage{soul}
\setuldepth{under}

\usepackage[letterpaper,margin=1in]{geometry}
\usepackage[parfill]{parskip}
\sloppy

\usepackage{soul}
\newcommand{\hlgray}[1]{{\sethlcolor{lightgray}\hl{#1}}}

\usepackage{float}
\usepackage[breakable, theorems, skins]{tcolorbox}

%spacing
\def\lstsp{\hspace{\labelsep}}
\def\tab{\indent\hspace{15pt}}

%new commands
\def\dx{\mathrm{d}x}
\def\Z{\mathbb{Z}}
\def\N{\mathbb{N}}
\def\R{\mathbb{R}}
\def\Q{\mathbb{Q}}
\def\C{\mathbb{C}}
\def\qedsymbol{$\blacksquare$}
\newcommand{\bem}{\begin{bmatrix}}
\newcommand{\enm}{\end{bmatrix}}


%calculus
\def\D{\mathrm{d}}
\def\dint{\displaystyle\int}
\newcommand{\diff}[3][]{\ensuremath{\frac{\D^{#1} #2}{\D #3^{#1}}}}
\newcommand\scalemath[2]{\scalebox{#1}{\mbox{\ensuremath{\displaystyle #2}}}} %https://tex.stackexchange.com/questions/60453/reducing-font-size-in-equation
\newcommand{\partials}[3][]{\ensuremath{\frac{\partial^{#1} {#2}}{\partial {#3}^{#1}}}}

\tcbset{
	exstyle/.style={enhanced, breakable, beforeafter skip balanced=10pt, coltitle=black, theorem style=plain, terminator sign={.\ \ \ }, fonttitle=\bfseries\upshape, fontupper=\upshape, blanker, borderline west={4pt}{-8pt}{orange!75!white}},
}

\newtcbtheorem[number within=subsection]{example}{Example}{exstyle}{ex}
\newtcbtheorem[number within=subsection]{examples}{Examples}{exstyle}{ex}

%img
\usepackage{import}
\usepackage{xifthen}
\usepackage{pdfpages}
\usepackage{transparent}

\newcommand{\incfig}[1]{%
    \def\svgwidth{\columnwidth}
    \import{./images/}{#1.pdf_tex}
}


%%% Theorem styles %%%
\theoremstyle{plain}
\newtheorem{theorem}{Theorem} % Enables you to use the theorem environment
\newtheorem{claim}[theorem]{Claim}
\newtheorem*{claim*}{Claim}
\newtheorem{proposition}[theorem]{Proposition}
\newtheorem*{proposition*}{Proposition}
\newtheorem{corollary}[theorem]{Corollary}
\newtheorem*{corollary*}{Corollary}
\newtheorem{lemma}[theorem]{Lemma}
\newtheorem*{lemma*}{Lemma}
\theoremstyle{definition}\newtheorem{definition}[theorem]{Definition}
\theoremstyle{definition}\newtheorem*{definition*}{Definition}


\title{Physics 137B Discussion 1}
\author{Walter Cheng}
\date{27 August, 2025}


\begin{document}
\maketitle

\section*{Exercise 0: Warm Up}
\begin{enumerate}[label=\alph*)]
	\item What is the momentum operator in the position representation in three dimensions?\\
		\textit{Hint:} What is the canonical commutation relation?\\

		answer
 
	\item From (a), what is $p^2 = \mathbf{p \cdot p}$ in three dimensions in the position representation?\\
		
		answer
	
	\item Starting from the time-dependent Schrödinger equation in three dimensions for a potential $V(\mathbf{x})$, derive the time-independent version.\\
		\textit{Hint:} Use separation of variables.\\

		answer

\end{enumerate}


\section*{Exercise 1: Rigged Hilbert Space}
A Hilbert space is mathematically defined as a \textit{complete} vector space with an inner product. A vector space with an inner product is \textbf{complete} if it includes not only all finite sums of vectors in a basis, but also all limits of convergent sequences, i.e.\ given a sequence $(v_n)$ of vectors in the Hilbert space, $v$ is the \textbf{limit} of the sequence if $\lim_{n\to\infty}\|v_n - v\| = 0$, where $\|v\| = \sqrt{v \cdot v}$.

\begin{enumerate}[label=\alph*)]
\item Consider a Hilbert space $\mathcal H$ that consists of all functions $\psi(x)$ such that
\[
\int_{-\infty}^{\infty} |\psi(x)|^2\,dx < \infty .
\]
Show that there are functions in $\mathcal H$ for which $\hat{x}\psi(x)=x\psi(x)$ is not in $\mathcal H$.\\

answer

\item Consider the function space $\Omega \subset \mathcal H$ which consists of all $\varphi(x)$ that satisfy the set of conditions
\[
\int_{-\infty}^{\infty} |\varphi(x)|^2 \,(1+|x|)^n\,dx < \infty,
\]
for any $n\in\{0,1,2,\dots\}$. Show that for any $\varphi(x)\in\Omega$, $\hat{x}\varphi(x)$ is also in $\Omega$. $\Omega$ is called the \textbf{nuclear space}. \emph{Hint:} Binomial theorem.\\

answer
	
\item The \textbf{extended} space $\Omega^\times$ consists of those functions $\chi(x)$ which satisfy
\[
(\chi,\varphi)=\int_{-\infty}^{\infty}\chi^{*}(x)\,\varphi(x)\,dx < \infty,
\]
for any $\varphi\in\Omega$, where $(\,\cdot\,,\,\cdot\,)$ is the inner product on $\mathcal H$. Which of the following functions belong to $\Omega$, to $\mathcal H$, and/or to $\Omega^\times$? \emph{Hints:} In order to sit in $\Omega$, functions must vanish faster than any power of $x$ as $|x|\to\infty$. Thus, as long as functions don’t diverge at $\infty$ more strongly than any power of $|x|$, they are in $\Omega^\times$.

\medskip
\noindent\textbf{Remark.} The collection $(\Omega, \mathcal H, \Omega^\times)$ is called ``rigged Hilbert space,'' and this is a rigorous way to include all the formalism (e.g.\ eigenvectors of position are delta functions, and hence can’t belong to an $L^2$ space) into the Hilbert space formulation of quantum mechanics. Note that $\Omega \subset \mathcal H \subset \Omega^\times$ (it’s easy to see this once you realize $\mathcal H = \mathcal H^\times$). For more details, see Ballentine \emph{Quantum Mechanics}, Chapter~1.

\begin{enumerate}[label=\roman*)]
\item $\sin(x)$\\

	answer

\item $\dfrac{\sin(x)}{x}$\\

	answer

\item $x^{2}\cos(x)$\\

	answer

\item $e^{-ax}, \; a > 0.$\\

	answer

\item $\dfrac{\ln(1+|x|)}{1+|x|}$\\

	answer

\item $e^{-x^{2}}$\\

	answer

\item $x^{4}e^{-|x|}$\\

	answer

\item $\delta(x-a)$ for $a$ real.\\

	answer
\end{enumerate}

\end{enumerate}

\section*{Exercise 2: Harmonic Oscillator}
Solve the eigenvalue problem for the 3-D isotropic, harmonic oscillator, whose hamiltonian is 
\[
H = \frac{p^2}{2m} + \frac{m\omega^2 x^2}{2}, \quad \text{where } p^2 = \mathbf{p}\cdot \mathbf{p}, \; x^2 = \mathbf{x}\cdot \mathbf{x}
\]
is the 3-D dot product.  
\emph{Hint:} There’s a way to do this without any calculations (if you remember the 1-D oscillator)!\\

answer

\section*{Exercise 3}
A particle of mass $m$ is placed in a finite spherical well
\[
V(r) = \begin{cases}
 -V_0, & r \leq a, \\
 0, & r \geq a .
\end{cases}
\]
Find the equation that quantizes the energy (you don’t need to solve it), by solving the radial Schrödinger equation with $\ell = 0$. Explain how you could solve this equation and obtain the energies. Show that there is no bound state if $V_0 a^2 < \pi^2 \hbar^2 / 8m$.  

\emph{Hint:} Recall that the radial Schrödinger equation is identical to the time-independent, 1-dimensional Schrödinger equation with the wavefunction replaced by $u(r) = rR(r)$ (where $\psi(r,\theta,\varphi) = R(r)\Theta(\theta)\Phi(\varphi)$) and potential
\[
V_{\text{eff}}(r) = V(r) + \frac{\hbar^2 \ell(\ell+1)}{2mr^2}.
\]\\

answer


\section*{Exercise 4: Spin Representations}
\begin{enumerate}[label=\alph*)]
\item Find the eigenvalues and eigenvectors of $S_z$.\\

answer

\item Do the same for $S_y$, and write them in terms of $\ket{\uparrow}$ and $\ket{\downarrow}$, the eigenvectors of $S_z$.\\

answer

\item For a system of two spin-$1/2$ particles, starting with the ``highest weight'' state $\ket{\uparrow \uparrow}$, find all the states in the triplet.  
\emph{Hint:} Apply the lowering operator.\\

answer

\item For a system of two spin-$1/2$ particles, are there any other states than the ones you found in (c)? If so, what are they? What is the action of $S_-$, $S_+$ on them?\\

answer

\item Describe how you would approach finding the Clebsch–Gordan coefficients for arbitrary spin systems.

answer

\end{enumerate}


\end{document}

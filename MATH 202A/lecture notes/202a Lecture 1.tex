\documentclass[11pt]{article}

\usepackage[utf8]{inputenc}
\usepackage{amsmath, amsthm, amssymb, amsfonts}
\usepackage{dsfont}
\usepackage{mathrsfs}
\usepackage{mathtools}
\usepackage{thmtools}
\usepackage{enumitem}
\usepackage{array}
\usepackage{subfigure}
\usepackage[all,arc]{xy}
\usepackage{tikz-cd}
\usepackage{xcolor}
\usepackage{graphicx}
\graphicspath{{./images/}}
\usepackage{wrapfig}
\usepackage{todonotes}
\usepackage[titletoc,toc,title]{appendix}
\usepackage[colorlinks = true,
            linkcolor = blue,
            urlcolor  = blue,
            citecolor = blue,
            anchorcolor = blue]{hyperref}
\usepackage{url}
\usepackage{soul}
\setuldepth{under}

\usepackage[letterpaper,margin=1in]{geometry}
\usepackage[parfill]{parskip}
\sloppy

\usepackage{soul}
\newcommand{\hlgray}[1]{{\sethlcolor{lightgray}\hl{#1}}}

\usepackage{float}
\usepackage[breakable, theorems, skins]{tcolorbox}

%spacing
\def\lstsp{\hspace{\labelsep}}
\def\tab{\indent\hspace{15pt}}

%new commands
\def\dx{\mathrm{d}x}
\def\Z{\mathbb{Z}}
\def\N{\mathbb{N}}
\def\R{\mathbb{R}}
\def\Q{\mathbb{Q}}
\def\C{\mathbb{C}}
\def\qedsymbol{$\blacksquare$}
\newcommand{\bem}{\begin{bmatrix}}
\newcommand{\enm}{\end{bmatrix}}


%calculus
\def\D{\mathrm{d}}
\def\dint{\displaystyle\int}
\newcommand{\diff}[3][]{\ensuremath{\frac{\D^{#1} #2}{\D #3^{#1}}}}
\newcommand\scalemath[2]{\scalebox{#1}{\mbox{\ensuremath{\displaystyle #2}}}} %https://tex.stackexchange.com/questions/60453/reducing-font-size-in-equation
\newcommand{\partials}[3][]{\ensuremath{\frac{\partial^{#1} {#2}}{\partial {#3}^{#1}}}}

\tcbset{
	exstyle/.style={enhanced, breakable, beforeafter skip balanced=10pt, coltitle=black, theorem style=plain, terminator sign={.\ \ \ }, fonttitle=\bfseries\upshape, fontupper=\upshape, blanker, borderline west={4pt}{-8pt}{orange!75!white}},
}

\newtcbtheorem[number within=subsection]{example}{Example}{exstyle}{ex}
\newtcbtheorem[number within=subsection]{examples}{Examples}{exstyle}{ex}

%img
\usepackage{import}
\usepackage{xifthen}
\usepackage{pdfpages}
\usepackage{transparent}

\newcommand{\incfig}[1]{%
    \def\svgwidth{\columnwidth}
    \import{./images/}{#1.pdf_tex}
}


%%% Theorem styles %%%
\theoremstyle{plain}
\newtheorem{theorem}{Theorem} % Enables you to use the theorem environment
\newtheorem{claim}[theorem]{Claim}
\newtheorem*{claim*}{Claim}
\newtheorem{proposition}[theorem]{Proposition}
\newtheorem*{proposition*}{Proposition}
\newtheorem{corollary}[theorem]{Corollary}
\newtheorem*{corollary*}{Corollary}
\newtheorem{lemma}[theorem]{Lemma}
\newtheorem*{lemma*}{Lemma}
\theoremstyle{definition}\newtheorem{definition}[theorem]{Definition}
\theoremstyle{definition}\newtheorem*{definition*}{Definition}


\title{Math 202a Lecture Notes\\ Lecture 1}
\date{28 August, 2025}
\author{Walter Cheng}

\begin{document}

\maketitle

\section{Conceptual Introduction}
A main theme of this course will be taking some classical space and through some expansion considering it a core to a larger modern space. A good example of this is the rational $\to$ real numbers relationship.

We have the rationals, which we can relate to a sieve; there are clear holes in it, but the rationals themselves are still countable and dense. We use a procedure for "completion" that will expand the rationals into the "solid line" of the reals.

Just a reminder,
\definition[Rationals] A \textbf{rational number} $q$ is represented as
\[q = \frac{m}{n}, \quad m \in \Z, n \in \N^+\]

This set forms a metric space with the following distance function,
\[d(q_1, q_2) = \frac{|m_1n_2 - m_2n_1|}{n_1n_2}\]

\section{Equivalence}

Our goal now is to express the irrationals as a limit of a sequence of rational numbers. First we'll define some stuff that will help with that.\\
\definition[Equivalence Relation] An \textbf{equivalence relation} is a set and operation $(S, \sim)$ that satisfies the following 3 axioms. We can think of $\sim \, \subseteq S \times S$. For $x, y, z \in S$,
\begin{enumerate}
	\item (reflexive) $x \sim x$
	\item (symmetric) $x \sim y \implies y \sim x$
	\item (transitive) $x \sim y \wedge y \sim z \implies x \sim z$
\end{enumerate}

We can partition $S$ into equivalence classes, where the equivalence class for some $s \in S$ is the set $\{x \in S : x \sim s\}$.

\section{Convergence}
Consider a sequence in $\Q$, $(x_n : n \in \N)$. We can define convergence of this sequence to some point $x$ by saying that $\lim_n |x_n - x| = 0$, but what if our limit lands outside of $\Q$? To keep things internal, we will instead use the idea of a Cauchy sequence.\\
\definition[Cauchy Sequence] A sequence is \textbf{Cauchy} if $\forall \epsilon > 0$, $\exists n_0$ s.t. $\forall n, m \geq n_0$,
\[d(x_n, x_m) < \epsilon\]

i.e. the points in the sequence eventually get very close together. This fixes our problem of defining irrational numbers as the limit of a sequence of rationals; is it true that
\[0.999\ldots = 1.000\ldots\]

using our notion of equivalence relation, we can say that
\[(x_n) \sim (y_n) \iff |x_n - y_n| \to 0\]

as well as express the distance between two irrational numbers,
\[d(x, y) = \lim_n|x_n - y_n|\]

This completes our definition of an irrational number internally (within $\Q$).

\section{Size of Sets}
\textit{Refer to section 2.2 of course notes}

We'll now consider intervals with an open left and closed right. Define the set of all finite disjoint unions of these to be $\mathcal{I}$. Formally,
\[\mathcal{I} = \bigcup_{i=1}^n (a_i, b_i]\]

where $n \in \N^+$ and $a_1 < b_1 < a_2 < \ldots < a_n < b_n$. Also permit $a_i, b_i \in \{-\infty, \infty\}$.

$\mathcal{I}$ is closed under complement, pairwise intersection, and finite intersection (shown by induction).

We define $l(I) = \sum_{i=1}^n (b_i - a_i) \in [0, \infty) \cup \{\infty\}$. Now we want to define the notion of distance.

We'll do this with symmetric difference:
\[A \triangle B = (A \setminus B) \cup (B \setminus A)\]

and then define $d(I_1, I_2) = l(I_1 \triangle I_2)$, for $I_1, I_2 \in \mathcal{I}$. It can be shown that $d$ is a metric on $\mathcal{I}$.

Now consider the closure of $\mathcal{I}$, $\overline{\mathcal{I}}$. Our aim is to make this a metric space. Let $I, J \in \overline{\mathcal{I}}$. We'll define distance like this:
\[d(I, J) = \lim_n d(I_n, J_n)\]

From this, we have a notion of measure:
\[l(I) = \lim_n l(I_n)\]

\section{Integration}
\textit{Section 2.3 of course notes}

From calculus, we understand the notion of integration as "area under the curve" $= \int_a^b f$. From analysis (Math 104) we know that we the idea of Riemann integration centers around a set of partitions $P$ and infimum and supremum integrals 
\[I^{\downarrow}(f,P) = \sum_{i=1}^n \inf\{f(x) : x \in [x_{i-1}, x_i]\}(x_i - x_{i-1})\]
and similarly for $I^{\uparrow}$.

From this, we define integrability:\\
\definition[Riemann Integrable] $f\colon [a, b] \to \R$ is \textbf{Riemann integrable} if $\exists I \in \R$ s.t. $\forall \epsilon > 0, \, \exists \delta > 0$ s.t. $\forall P$ with $\operatorname{mesh}(P) < \delta$,
\[|I^{\uparrow}(f, P) - I| < \epsilon\]

Where $\operatorname{mesh}(P)$ is the length of the longest interval in $P$.

The integral-defined function
\[d(f, g) = \int|f-g|\]
is a metric on the space of continuous functions called the $L^1$ norm. We can define pointwise convergence in $L^1$.

Now consider
\[f_n \to \mathds{1}_{[1/2, 1]}\]

where $f_n\colon [0,1] \to [0,1]$ are pointwise decreasing. We can have the problem where each $f_n$ is Riemann integrable, but
\[f = \lim_n f_n\]
is not. Thus, we wish to broaden our integration to Lebesgue theory.

An important example to highlight is the integral of the indicator function. What is
\[\int_0^1 \mathds{1}_{\Q}\]
where $\mathds{1}_{\Q} = \begin{cases}
	1 & \text{ if $x \in \Q$}\\
	0 & \text{ else}
\end{cases}$

Riemann integration says this integral doesn't exist ($\Q$ is dense in $\R$). But intuitively we think it should really be $0$.

\end{document}

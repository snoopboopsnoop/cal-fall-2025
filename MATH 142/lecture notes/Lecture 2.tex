\documentclass[11pt]{article}


\usepackage[utf8]{inputenc}
\usepackage{amsmath, amsthm, amssymb, amsfonts}
\usepackage{mathrsfs}
\usepackage{mathtools}
\usepackage{thmtools}
\usepackage{enumitem}
\usepackage{array}
\usepackage{subfigure}
\usepackage[all,arc]{xy}
\usepackage{tikz-cd}
\usepackage{xcolor}
\usepackage{graphicx}
\usepackage{braket}
\graphicspath{{./images/}}
\usepackage{wrapfig}
\usepackage{todonotes}
\usepackage[titletoc,toc,title]{appendix}
\usepackage[colorlinks = true,
            linkcolor = blue,
            urlcolor  = blue,
            citecolor = blue,
            anchorcolor = blue]{hyperref}
\usepackage{url}
\usepackage{soul}
\setuldepth{under}

\usepackage[letterpaper,margin=1in]{geometry}
\usepackage[parfill]{parskip}
\sloppy

\usepackage{soul}
\newcommand{\hlgray}[1]{{\sethlcolor{lightgray}\hl{#1}}}

\usepackage{float}
\usepackage[breakable, theorems, skins]{tcolorbox}

%spacing
\def\lstsp{\hspace{\labelsep}}
\def\tab{\indent\hspace{15pt}}

%new commands
\def\dx{\mathrm{d}x}
\def\Z{\mathbb{Z}}
\def\N{\mathbb{N}}
\def\R{\mathbb{R}}
\def\Q{\mathbb{Q}}
\def\C{\mathbb{C}}
\def\qedsymbol{$\blacksquare$}
\newcommand{\bem}{\begin{bmatrix}}
\newcommand{\enm}{\end{bmatrix}}


%calculus
\def\D{\mathrm{d}}
\def\dint{\displaystyle\int}
\newcommand{\diff}[3][]{\ensuremath{\frac{\D^{#1} #2}{\D #3^{#1}}}}
\newcommand\scalemath[2]{\scalebox{#1}{\mbox{\ensuremath{\displaystyle #2}}}} %https://tex.stackexchange.com/questions/60453/reducing-font-size-in-equation
\newcommand{\partials}[3][]{\ensuremath{\frac{\partial^{#1} {#2}}{\partial {#3}^{#1}}}}

\tcbset{
	exstyle/.style={enhanced, breakable, beforeafter skip balanced=10pt, coltitle=black, theorem style=plain, terminator sign={.\ \ \ }, fonttitle=\bfseries\upshape, fontupper=\upshape, blanker, borderline west={4pt}{-8pt}{orange!75!white}},
}

\newtcbtheorem[number within=subsection]{example}{Example}{exstyle}{ex}
\newtcbtheorem[number within=subsection]{examples}{Examples}{exstyle}{ex}

%img
\usepackage{import}
\usepackage{xifthen}
\usepackage{pdfpages}
\usepackage{transparent}

\newcommand{\incfig}[1]{%
    \def\svgwidth{\columnwidth}
    \import{./images/}{#1.pdf_tex}
}


%%% Theorem styles %%%
\theoremstyle{plain}
\newtheorem{theorem}{Theorem} % Enables you to use the theorem environment
\newtheorem{claim}[theorem]{Claim}
\newtheorem*{claim*}{Claim}
\newtheorem{proposition}[theorem]{Proposition}
\newtheorem*{proposition*}{Proposition}
\newtheorem{corollary}[theorem]{Corollary}
\newtheorem*{corollary*}{Corollary}
\newtheorem{lemma}[theorem]{Lemma}
\newtheorem*{lemma*}{Lemma}
\theoremstyle{definition}\newtheorem{definition}[theorem]{Definition}
\theoremstyle{definition}\newtheorem*{definition*}{Definition}


\title{Math 142 Lecture Notes\\ Lecture 2}
\date{29 August, 2025}
\author{Walter Cheng}

\begin{document}

\maketitle

\section{Announcements}
\begin{itemize}
	\item Midterm 1: Monday, October 6
	\item Midterm 2: Friday, November 14
	\item Homework logistics: Each lecture Professor Rieffel will assign reading/exercises, they will be due on Fridays at 11:59pm. No late homework will be accepted, but the lowest 4 scores will be dropped.
	\item \textbf{Reading Assignment}: Read \S 12 and page 119
	\item \textbf{Assigned HW Problems}: Exercises 2 \& 3 on page 83 
\end{itemize}

\section{Generalizing Continuity}
Recall from Lecture 1: Given metric spaces $(X, d^X), (Y, d^Y)$ and $f\colon X \to Y$,\\
\definition[Continuity at a point] $f$ is continuous at $x_0$ if for any open $O \subseteq Y$ with $f(x_0) \in O$ there is an open set $U \subseteq X$ containing $x_0$ such that $U \subseteq f^{-1}(O)$. 

We can broaden this definition to continuity of a function:\\
\definition[Continuity] $f$ is continuous if it is continuous at every $x \in X$. Formally, for every $x \in X$ and $O \subseteq Y$ with $f(x) \in O$, there is an open $U \subseteq X$ s.t. $U \subseteq f^{-1}(O)$.

In general terms we can say that $f$ is continuous if for any open $O$ in $Y$ the preimage of $O$ is open in $X$. These definitions are equivalent because any point $x \in f^{-1}(O)$ corresponds to a point $x_0 \in O$, and by definition of continuity there is an open set around $x$.

\section{Topology}
Consider a collection $\mathcal{T}^X$ of open sets in $(X, d^X)$. It has the following properties:
\begin{enumerate}
	\item $X, \emptyset \in \mathcal{T}^X$
	\item Any arbitrary (possibly infinite) union of a collection of open sets is open.\\
		If $x$ is in the union then it is part of one of the open sets, therefore it must have an open ball around it which is also in the union.
	\item Any \textbf{finite} intersection of open sets is open.\\
		If $x$ is in the finite intersection of open sets then it must be in all open sets. Therefore for the $j$th open set there is a $B(x, r_j)$ in the open set. Let $r = \min\{r_0\ldots r_n\}$, then $B(x, r) \subseteq \bigcap_{i=1}^{n} O_i$.
\end{enumerate}

In a kind of roundabout way, we can abstract these properties to a structure known as a topology.\\
\definition[Topology] Given a set $X$, a \textbf{topology} $\mathcal{T}$ on $X$ is a collection of subsets of $X$ that satisfies the properties above. Rewritten, those are
\begin{enumerate}
	\item $X, \emptyset \in \mathcal{T}$
	\item Any arbitrary union of elements in $\mathcal{T}$ is in $\mathcal{T}$
	\item Any finite intersection of elements in $\mathcal{T}$ is in $\mathcal{T}$
\end{enumerate}

Note that there is no metric involved in this definition.

\section{Comparing Topologies}
If $\mathcal{T}_1$ and $\mathcal{T}_2$ are topologies on $X$, we say that $\mathcal{T}_1$ is \textbf{stronger/finer/bigger} than $\mathcal{T}_2$ if $\mathcal{T}_2 \subseteq \mathcal{T}_1$. 

Conversely, we say $\mathcal{T}_1$ is \textbf{weaker/coarser/smaller} than $\mathcal{T}_2$ if $\mathcal{T}_1 \subseteq \mathcal{T}_2$.

There is a biggest topology on $X$, namely the topology consisting of all subsets of $X$, the power set of $X$. This topology corresponds to a metric, where $d(x_1, x_2) = 1$ if $x_1 \neq x_2$ and $0$ otherwise. This is called the \textbf{discrete} topology (use balls of radius $1/2$ to show openness).

There is also a smallest topology, consisting of $\{X, \emptyset\}$, called the \textbf{trivial} or \textbf{indiscrete} topology. Given that $|X| \geq 2$, this does not come from a metric.

\end{document}
